% ------------------------------------------------------------------------------ %

  \documentclass[a4paper,11pt]{report}

% ------------------------------------------------------------------------------- %
%LANGUAGE PACKS%

  \usepackage[english]{babel} 
 % ------------------------------------------------------------------------------- %
%FONTS%    
  \usepackage[utf8x]{inputenc}
  
  \usepackage{amsmath}
  \usepackage{multirow}
  \usepackage{enumerate}
  \usepackage{verbatim}
  \usepackage{listings}
% ------------------------------------------------------------------------------- %
%LAYOUT%  

\usepackage{fullpage}

 \tolerance=1
 \emergencystretch=\maxdimen
 \hyphenpenalty=10000
 \hbadness=10000
 \newcommand{\HRule}{\rule{\linewidth}{0.5mm}}
 
 \usepackage[pdftex]{graphicx} 

 
 \usepackage{hyperref}

 \hypersetup{
    colorlinks,
    citecolor=black,
    filecolor=black,
    linkcolor=black,
    urlcolor=black
}
\lstset {breaklines=true,
extendedchars=false,
showstringspaces=false}

% ------------------------------------------------------------------------------- %

% ------------------------------------------------------------------------------- %


\begin{document}

\begin{titlepage}

\begin{center}

\includegraphics{images/UvA-logo-2a.png}~\\[1cm]

\textsc{\LARGE System and Network Engineering\\ OS3 Group}\\[1.5cm]

\HRule \\

{ \huge \bfseries Essential Skills\\Latex Assignment}

\HRule \\[1cm]

\large{Diana Rusu} \\
\large{Alex Stavroulakis}\\
\large{Nick Triantafylldis}\\[12cm]

\today
\end{center}
\end{titlepage}

\tableofcontents

\chapter* {Introduction}
\addcontentsline{toc}{chapter}{Introduction}

This is our report in latex.

\chapter {Build Tools: Scons}


\chapter{XQuery Definition}

\begin{center}
\includegraphics{images/xquery_image.jpeg}~\\[1cm]

\end{center}

\paragraph{}
\textbf{Xquery} is to XML formatted data what SQL is to
relational databases.\\

\begin{itemize}
  \item \textbf{Xquery} is a specification from \textit{W3C}, appeared in 2007 and became a \textit{W3C} reccomendation.
  \item Current version 3.0.
  \item It is a language that uses queries to extract data from XML file or XML-like files.
  \item It contains a superset of Xpath expression syntax.
\end{itemize}
 
 Also, \\
 
 \begin{enumerate}
  \item \textbf{XQuery} is extensible.
  \item It contains many functions such as mathematical operations.
  \item If a function is needed and is not found, it can be added or programmed.
  \item Vendor-specific extensions allow other formats than XML to be queried.
\end{enumerate}


\paragraph{}
This presentation was a \emph{collaboration} between these groups: \\
 
\begin{center}
 \begin{tabular}{| l | c | c | c |}
 \hline \textbf{Group} & \textbf{Student} & \textbf{Link} \\
 \hline 
 4 & Alex Stavroulakis & \href{https://www.os3.nl/2014-2015/students/alexandros_stavroulakis/es}{Wiki}\\
 \hline 
 5 & Guido Kroon & \href{https://www.os3.nl/2014-2015/students/guido_kroon/es/assignments1}{Wiki} \\ 
 \hline 
 6 & Rophrimardho & \href{https://www.os3.nl/2014-2015/students/rohprimardho/es/homework_1.3}{Wiki} \\ 
 \hline 
 \end{tabular} 
\end{center}
 

\chapter{RegEx Text Directed Engine}

\begin{center}
\includegraphics{images/regexp_logo.png}~\\[1cm]
\end{center}

\section{Types of Engines}
\begin{flushleft}
Two basic engine types reflect a fundamental difference in algorithms available for applying a regular expression to a string. The \textbf{NFA} (Non Deterministic Finite Automations) "regex-directed" engines and the \textbf{DFA} (Deterministic Finite Automations) "text-directed" engines.
\end{flushleft}

\section{Text-directed Engines}
\begin{center}
According to Jeffrey E. F. Friedl’s book "Mastering Regular Expressions" a DFA engine, each character scanned from the text controls the engine.
\end{center}

\section{How does it Work}
\begin{flushright}
It walks through the input string attempting all permutations of the regular expression before reading the next character in the string. It does not use backtracking
\end{flushright}

\section{Where is it used}
\begin{itemize}
  \item egrep
  \item awk
  \item MySQL etc\ldots
\end{itemize}

\paragraph{}
This presentation was a \emph{collaboration} between these groups: \\

\begin{center}
 \begin{tabular}{| l | c | c | c |}
 \hline \textbf{Group} & \textbf{Student} & \textbf{Link} \\
 \hline 
 4 & Alex Stavroulakis & \href{https://www.os3.nl/2014-2015/students/alexandros_stavroulakis/es}{Wiki}\\
 \hline 
 5 & Dragos Barosan & \href{https://www.os3.nl/2014-2015/students/dragos_barosan/es/week2#homework_3}{Wiki} \\ 
 \hline 
 6 & Carlo Rengo & \href{https://www.os3.nl/2014-2015/students/carlo_rengo/es/homewrk_3}{Wiki} \\ 
 \hline 
 \end{tabular} 
\end{center}

\end{document}





